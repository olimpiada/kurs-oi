\documentclass{spiral-kurs}
\def\title{Wicemistrz}
\def\id{wic}
\def\TL{1~s}
\def\ML{256~MB}
\begin{document}
\makeheader
%
    W turnieju podwórkowym w pewną grę zespołową wzięło udział $n$ drużyn.
    Każda z drużyn zdobyła łącznie inną liczbę punktów.
    Dzieci cały czas śledziły, która drużyna prowadziła w turnieju, więc
    wiedzą już, która drużyna wygrała cały turniej.
    Teraz dzieci postanowiły sprawdzić, która drużyna zajęła drugie miejsce
    w~turnieju.
    Napisz program, który pomoże im to określić.

    \section{Wejście}
    W pierwszym wierszu wejścia znajduje się jedna liczba całkowita $n$
    ($2 \le n \le 1000$), oznaczająca liczbę dzieci.
    W drugim wierszu znajduje się $n$ liczb całkowitych $p_1,\ldots,p_n$
    ($1 \le p_i \le 1\,000\,000$), oddzielonych spacjami.
    Liczby te oznaczają liczby punktów zdobyte przez poszczególne drużyny w turnieju.
    Liczby $p_i$ będą parami różne.

    \section{Wyjście}
    Twój program powinien wypisać jedną liczbę całkowitą: liczbę punktów zdobytą
    przez drużynę, która została wicemistrzem turnieju, czyli drugą od góry wartość
    w ciągu $p_1,\ldots,p_n$.

    \example{0}


  \end{document}

\documentclass{spiral-kurs}
\def\title{Łańcuszek}
\def\id{lan}
\def\TL{1~s}
\def\ML{256~MB}
\begin{document}
\makeheader
%
    Niedawno na portalach społecznościowych popularny był następujący łańcuszek:

    \medskip
    {\it
      \noindent
      Sprawdźcie, czy Wasz numer komórki zdradza Wasz wiek! U mnie i rodziny się sprawdziło. Niesamowite:)
      \begin{enumerate}
        \item Weź ostatnią cyfrę numeru komórki i pomnóż razy 2 (u mnie $2 \cdot 2=4$)
        \item Do wyniku dodaj 5 ($4+5=9$)
        \item Pomnóż razy 50 ($9 \cdot 50=450$)
        \item Dodaj 1764 ($450+1764=2214$)
        \item Odejmij swój rok urodzenia ($2214-1998=216$!!!)
      \end{enumerate}
      Pierwsza cyfra otrzymanego wyniku to ostatnia cyfra twojego numeru komórki, a dwie ostatnie cyfry to twój rocznikowy wiek!!!

      \medskip
      \noindent
      Zgadza się?
    }

    \medskip
    \noindent
    Napisz program, który pomoże internautom sprawdzić, czy również w ich przypadku
    wynik operacji w łańcuszku zdradza ich wiek.

    \section{Wejście}
    Twój program powinien wczytać dwie liczby całkowite, oddzielone spacją dziewięciocyfrowy numer telefonu
    oraz rok urodzenia internauty (z przedziału od $1920$ do $2007$).

    \section{Wyjście}
    Twój program powinien wypisać liczbę otrzymaną w wyniku operacji 1--5 z łańcuszka.

    \example{0}

    \medskip
    \noindent
    W chwili wolnej od pisania programu możesz spróbować rozwikłać \emph{magiczną}
    sztuczkę wykorzystaną w łańcuszku i odkryć, czemu -- poza zaskoczeniem odbiorców
    -- mógł służyć ten łańcuszek.


  \end{document}

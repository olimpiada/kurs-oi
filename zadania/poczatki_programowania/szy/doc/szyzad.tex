\documentclass{spiral-kurs}
\def\title{Szyfr Cezara (*)}
\def\id{szy}
\def\TL{1~s}
\def\ML{256~MB}
\begin{document}
\makeheader
%
  W tym zadaniu Twój program będzie szyfrował i odszyfrowywał dane, używając algorytmu,
  który był znany już w czasach Juliusza Cezara.
  Szyfrowanie tekstu polega na szyfrowaniu kolejnych jego liter (pozostałe znaki pozostawiamy bez zmian).
  Każda litera zostaje zamieniona w $k$-tą następną w alfabecie
  ($k$ jest stałą szyfru), przy czym jeżeli taka nie istnieje (wychodzimy za \texttt{z}), to
  odliczanie jest kontynuowane z powrotem od \texttt{a}.
  Szyfrowanie zachowuje wielkość liter (tj.\ małe przechodzą na małe, a wielkie na wielkie).
  Zakładamy, że w~tekście występują jedynie litery alfabetu angielskiego (małe i wielkie)
  oraz znaki interpunkcyjne (bez spacji).

  Dla przykładu, jeżeli $k=5$, to małe litery tekstu przechodzą na małe litery szyfru według poniższej tabelki:

  \medskip
  \noindent
  {\small
    \begin{center}
  \begin{tabular}{|c|c|c|c|c|c|c|c|c|c|c|c|c|c|c|c|c|c|c|c|c|c|c|c|c|c|c|}
    \hline
    litera &
    \texttt{a} & \texttt{b} & \texttt{c} & \texttt{d} & \texttt{e} & \texttt{f} & \texttt{g} & \texttt{h} & \texttt{i} & \texttt{j} & \texttt{k} & \texttt{l} & \texttt{m} & \texttt{n} & \texttt{o} & \texttt{p} & \texttt{q} & \texttt{r} & \texttt{s} & \texttt{t} & \texttt{u} & \texttt{v} & \texttt{w} & \texttt{x} & \texttt{y} & \texttt{z} \\\hline
    szyfr &
    \texttt{f} & \texttt{g} & \texttt{h} & \texttt{i} & \texttt{j} & \texttt{k} & \texttt{l} & \texttt{m} & \texttt{n} & \texttt{o} & \texttt{p} & \texttt{q} & \texttt{r} & \texttt{s} & \texttt{t} & \texttt{u} & \texttt{v} & \texttt{w} & \texttt{x} & \texttt{y} & \texttt{z} & \texttt{a} & \texttt{b} & \texttt{c} & \texttt{d} & \texttt{e} \\\hline
  \end{tabular}
\end{center}
}

  \medskip
  \noindent
  Wielkie litery tekstu przechodzą na wielkie litery szyfru zgodnie z tą samą regułą.

  Napisz program, który wczyta tekst do zaszyfrowania lub odszyfrowania i stałą $k$
  i zaszyfruje lub odszyfruje ten tekst, w zależności od polecenia.

  \section{Wejście}
  Pierwszy wiersz wejścia zawiera jedną liczbę 1 lub 2.
  Liczba 1 oznacza ,,szyfruj'', a liczba 2 -- ,,odszyfruj''.
  Drugi wiersz wejścia zawiera stałą $k$ ($1\le k\le 25$).
  Trzeci i ostatni wiersz wejścia zawiera tekst, złożony wyłącznie z liter (małych bądź dużych) i/lub znaków interpunkcyjnych (bez spacji).
  Tekst będzie zawierał co najmniej jeden znak i co najwyżej $10\,000$ znaków.

  \section{Wyjście}
  Pierwszy i jedyny wiersz wyjścia powinien zawierać tekst po zaszyfrowaniu bądź odszyfrowaniu.

    \example{0}

    \noindent
    a dla danych wejściowych:

    \includefile{../in/\id 0a.in}

    \noindent
    poprawnym wynikiem jest:

    \includefile{../out/\id 0a.out}

    \noindent
    natomiast dla danych wejściowych:

    \includefile{../in/\id 0b.in}

    \noindent
    poprawnym wynikiem jest:

    \includefile{../out/\id 0b.out}


  \end{document}

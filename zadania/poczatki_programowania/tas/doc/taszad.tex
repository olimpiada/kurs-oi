\documentclass{spiral-kurs}
\def\title{Taśma (*)}
\def\id{tas}
\def\TL{3~s}
\def\ML{256~MB}
\begin{document}
\makeheader
%
      Jaś przypadkowo znalazł w domu bardzo długą taśmę.
      Bez chwili namysłu napisał na taśmie pewien ciąg liczb całkowitych dodatnich.
      Teraz chciałby znaleźć w tym ciągu dwie najdalej od siebie położone
      różne liczby.
      Zakładamy, że odległość między sąsiednimi liczbami to 1, między liczbami
      posiadającymi wspólnego sąsiada to 2 itd.

    \section{Wejście}
      Pierwszy wiersz wejścia zawiera jedną liczbę całkowitą $n$
      ($1\le n\le 500\,000$), oznaczającą długość sekwencji liczb
      zapisanej przez Jasia na taśmie.
      Drugi wiersz zawiera ciąg $n$ liczb całkowitych $a_i$
      ($1\le a_i\le 1\,000\,000\,000$), oddzielonych spacjami.

    \section{Wyjście}
      Jeżeli w podanym na wejściu ciągu
      nie ma żadnej pary różnych liczb, to $i$-ty wiersz powinien
      zawierać jedno słowo ,,\texttt{BRAK}''.
      W przeciwnym przypadku w $i$-tym wierszu powinna znajdować się
      jedna liczba całkowita, równa odległości między najdalszą
      parą różnych liczb w ciągu.

    \example{0}

    \medskip
    \noindent
    \textbf{Wyjaśnienie do przykładu:} najdalszymi różnymi
    liczbami w sekwencji są m.in.\ pierwsza (czyli 2) i siódma (czyli 5).

    \bigskip
    \noindent
    Natomiast dla danych wejściowych:

    \includefile{../in/\id 0a.in}

    \noindent
    poprawnym wynikiem jest:

    \includefile{../out/\id 0a.out}

  \end{document}

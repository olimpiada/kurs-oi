\documentclass{spiral-kurs}
\def\title{Koło}
\def\id{kol}
\def\TL{1~s}
\def\ML{256~MB}
\begin{document}
\makeheader
%
  Zadaniem Twojego programu będzie obliczenie pola i obwodu
  koła o zadanym promieniu.
  Interesuje nas wynik zaokrąglony do trzech cyfr po kropce dziesiętnej.

  \section{Wejście}
  Na wejściu znajduje się jedna liczba rzeczywista $r$
  ($0 < r \le 1000$), będąca długością promienia koła.

  \section{Wyjście}
  W pierwszym wierszu należy wypisać pole koła o promieniu
  $r$ z dokładnością do 3 miejsc po kropce dziesiętnej, a~w~drugim wierszu należy wypisać długość obwodu tego koła również
  z dokładnością do trzech miejsc po kropce.

  \example{0}

  \medskip
  \noindent
  \textbf{Wskazówka:} W pliku nagłówkowym \texttt{cmath} dostępne jest dobre przybliżenie
  liczby $\pi$ (stała \texttt{M\_PI}).
  Możesz też użyć własnego przybliżenia liczby $\pi$; pamiętaj jednak, że powinno być możliwie dokładne, tak
  aby wynik dla koła o maksymalnym możliwym w tym zadaniu promieniu był poprawny z dokładnością do 3 cyfr po kropce.


  \end{document}

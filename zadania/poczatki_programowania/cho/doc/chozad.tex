\documentclass{spiral-kurs}
\def\title{Choinka}
\def\id{cho}
\def\TL{1~s}
\def\ML{256~MB}
\begin{document}
\makeheader
%
  Na koniec zadanie świąteczne.
  Twoim zadaniem będzie wypisanie ,,choinki'' o zadanym rozmiarze za pomocą znaków '\texttt{*}'.

  \section{Wejście}
  W pierwszym i jedynym wierszu wejścia znajduje się
  jedna liczba całkowita $n$ ($2 \le n \le 20$) oznaczająca rozmiar choinki.

  \section{Wyjście}
  Wyjście powinno zawierać $n + 2$ wierszy.
  Pierwsze $n$ wierszy powinno zawierać górną część choinki,
  a~w~następnych dwóch wierszach powinien być pień choinki.

  Najlepiej opis choinki wyjaśnia przykład, ale dla formalności wyjaśniamy, jak
  jest ona zbudowana.
  W~pierwszym wierszu opisu górnej części choinki powinna być jedna gwiazdka,
  a w każdym kolejnym -- o dwie gwiazdki więcej.
  Ostatni wiersz opisu górnej części choinki (ten o numerze $n$) nie powinien rozpoczynać
  się od żadnych spacji.
  Ostatnie dwa wiersze powinny zawierać po jednej gwiazdce (pień choinki).
  Wszystkie wiersze powinny rozpoczynać się taką liczbą spacji, aby choinka
  miała pionową oś symetrii.

  Przypominamy, że spacje znajdujące się na końcu wiersza są ignorowane
  przez system sprawdzający, więc nie trzeba ich tam wypisywać.


    \example{0}


  \end{document}

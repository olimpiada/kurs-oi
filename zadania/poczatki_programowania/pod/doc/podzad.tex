\documentclass{spiral-kurs}
\def\title{Podzielne}
\def\id{pod}
\def\TL{1~s}
\def\ML{256~MB}
\begin{document}
\makeheader
%
  W tym zadaniu Twój program powinien obliczyć, ile liczb z przedziału od $a$ do $b$
  jest podzielnych przez $k$.

  \section{Wejście}
  Pierwszy i jedyny wiersz wejścia zawiera trzy liczby całkowite $a$, $b$, $k$
  ($1 \leq a \leq b \leq 2 \cdot 10^9$, $1 \leq k \leq 2 \cdot 10^9$), oddzielone spacjami,
  oznaczające odpowiednio początek i koniec przedziału oraz liczbę, przez którą dzielimy.

  \section{Wyjście}
  Wyjście powinno zawierać jedną liczbę całkowitą, równą liczbie liczb z przedziału $[a,b]$,
  które dzielą się przez $k$.

    \example{0}

    \noindent
    a dla danych wejściowych:

    \includefile{../in/\id 0a.in}

    \noindent
    poprawnym wynikiem jest:

    \includefile{../out/\id 0a.out}

  \end{document}

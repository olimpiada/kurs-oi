\documentclass{spiral-kurs}
\def\title{Dwa markety}
\def\id{dwa}
\def\TL{1~s}
\def\ML{256~MB}
\begin{document}
\makeheader
%
     Chcesz zrobić zakupy.
     Wiesz dokładnie, jakie produkty chcesz kupić.
     Sprawdziłeś już w Internecie ceny każdego z produktów we wszystkich
     okolicznych marketach.
     Masz czas pojechać do co najwyżej dwóch marketów i~łącznie chcesz
     w nich kupić po jednym egzemplarzu każdego produktu.
     Jak to zrobić najtaniej?

    \section{Wejście}
      Pierwszy wiersz wejścia zawiera dwie liczby całkowite $n$ oraz $m$
      ($2\le n,m\le 100$) oddzielone spacją, oznaczające liczbę marketów
      oraz liczbę produktów, które chcesz kupić.
      Każdy z kolejnych $n$ wierszy zawiera po $m$ liczb całkowitych
      z zakresu od 1 do $1000$.
      Pierwszy wiersz zawiera ceny kolejnych produktów w pierwszym markecie,
      drugi -- ceny kolejnych produktów w drugim markecie itd.

    \section{Wyjście}
      Twój program powinien wypisać jedną liczbę całkowitą: minimalny koszt
      zakupu wszystkich potrzebnych produktów w co najwyżej dwóch marketach.

    \example{0}

    \medskip
    \noindent
    \textbf{Wyjaśnienie do przykładu:}
    Najlepiej pojechać do pierwszego i drugiego marketu.
    W pierwszym kupujemy drugi i trzeci produkt (koszt $3+7$),
    a w drugim pierwszy i czwarty (koszt $2+6$).


  \end{document}

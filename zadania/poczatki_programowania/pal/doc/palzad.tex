\documentclass{spiral-kurs}
\def\title{Palindrom}
\def\id{pal}
\def\TL{1~s}
\def\ML{256~MB}
\begin{document}
\makeheader
%
  Palindrom (gr.\ \emph{palindromeo} -- biec z powrotem) to słowo brzmiące tak samo czytane od lewej do prawej i~od prawej do lewej.
  Przykładami palindromów w języku polskim są \texttt{anna} i \texttt{kajak}\footnote{
    Ciekawe przykłady palindromów w różnych językach można znaleźć na stronie
    \texttt{http://pl.wikipedia.org/wiki/Palindrom}
  }.
  W tym zadaniu chcemy sprawdzić, czy wczytane słowo jest palindromem.

  \section{Wejście}
  Na wejściu znajduje się jedno słowo złożone z małych liter alfabetu angielskiego.
  Długość słowa jest dodatnia i nie przekracza $1\,000\,000$.

  \section{Wyjście}
  Pierwszy i jedyny wiersz wyjścia powinien zawierać jedno słowo
  \texttt{TAK} lub \texttt{NIE}, stanowiące odpowiedź na pytanie,
  czy słowo podane na wejściu jest palindromem.


    \example{0}

    \noindent
    a dla danych wejściowych:

    \includefile{../in/\id 0a.in}

    \noindent
    poprawnym wynikiem jest:
    
    \includefile{../out/\id 0a.out}

  \end{document}

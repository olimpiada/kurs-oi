\documentclass{spiral-kurs}
\def\title{Jakie to działanie?}
\def\id{jak}
\def\TL{1~s}
\def\ML{256~MB}
\begin{document}
\makeheader
%
    W tym zadaniu Twój program powinien rozwiązać łamigłówkę postaci:
    $$a\ \mathtt{?}\ b\ \mathtt{==}\ c$$
    W tej łamigłówce $a$, $b$ i $c$ będą danymi liczbami całkowitymi nieujemnymi,
    a Twoim zadaniem jest powiedzieć, jakie działanie należy wstawić zamiast znaku zapytania,
    aby zachodziła równość.
    Masz do dyspozycji standardowe działania arytmetyczne dostępne w języku C++:
    \begin{itemize}
      \item \texttt{+} (dodawanie),
      \item \texttt{-} (odejmowanie),
      \item \texttt{*} (mnożenie),
      \item \texttt{/} (dzielenie z resztą),
      \item \texttt{\%} (reszta z dzielenia).
    \end{itemize}

    \section{Wejście}
    Pierwszy i jedyny wiersz wejścia zawiera trzy liczby całkowite $a$, $b$, $c$
    ($0 \le a, b, c \le 1000$), oddzielone spacjami.

    \section{Wyjście}
    Twój program powinien wypisać wiersz z rozwiązaniem łamigłówki, zastępując
    znak zapytania odpowiednim działaniem.
    Znak działania i znaki równości powinny być otoczone spacjami.

    Gwarantujemy, że dla każdej łamigłówki będzie co najmniej jedno rozwiązanie.
    Jeśli łamigłówka ma więcej niż jedno rozwiązanie, Twój program powinien
    wypisać \textbf{wszystkie} z nich, w osobnych wierszach, \textbf{według kolejności
    działań wypisanej powyżej}.

    \example{0}

    \noindent
    a dla danych wejściowych:

    \includefile{../in/\id 0a.in}

    \noindent
    poprawnym wynikiem jest:

    \includefile{../out/\id 0a.out}

    \noindent
    \textbf{Uwaga:} Pamiętaj, że nie wolno dzielić przez zero!

  \end{document}

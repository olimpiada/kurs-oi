\documentclass{spiral-kurs}
\def\title{Na odwrót}
\def\id{nao}
\def\TL{1~s}
\def\ML{256~MB}
\begin{document}
\makeheader
%
  Zadaniem Twojego programu będzie wczytanie trzech liczb całkowitych
  i wypisanie ich w takiej samej kolejności oraz w kolejności odwrotnej.

  \section{Wejście}
  W jedynym wierszu wejścia znajdują się trzy liczby całkowite oddzielone spacjami:
  $a$, $b$, $c$ ($-1000 \le a,b,c \le 1000$).

  \section{Wyjście}
  W pierwszym wierszu należy wypisać podane liczby w kolejności wczytania: $a$, $b$, $c$.
  W drugim wierszu należy wypisać podane liczby w kolejności odwrotnej do
  kolejności wczytania: $c$, $b$, $a$.
  W obu wierszach liczby powinny być rozdzielane pojedynczymi spacjami.

    \example{0}


  \end{document}

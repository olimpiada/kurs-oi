\documentclass{spiral-kurs}
\def\title{Prostopadłościan}
\def\id{pro}
\def\TL{1~s}
\def\ML{256~MB}
\begin{document}
\makeheader
%
  Zadaniem Twojego programu będzie obliczenie objętości i pola powierzchni prostopadłościanu
  o zadanych wymiarach.

  \section{Wejście}
  Jedyny wiersz wejścia zawiera trzy liczby całkowite dodatnie $a$, $b$, $c$
  ($1 \le a,b,c < 500\,000\,000$) oddzielone spacjami.
  Oznaczają one trzy wymiary prostopadłościanu, czyli długości trzech prostopadłych krawędzi.

  \section{Wyjście}
  W pierwszym wierszu należy wypisać objętość prostopadłościanu o
  krawędziach długości $a$, $b$, $c$.
  W drugim wierszu należy wypisać pole powierzchni tego prostopadłościanu.
  Możesz założyć, że zarówno pole powierzchni jak i objętość nie przekroczy
  $2\,000\,000\,000$.

    \example{0}


  \end{document}

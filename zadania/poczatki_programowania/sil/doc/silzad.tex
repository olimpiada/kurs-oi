\documentclass{spiral-kurs}
\def\title{Silnia}
\def\id{sil}
\def\TL{1~s}
\def\ML{256~MB}
\begin{document}
\makeheader
%
    \textbf{Silnią} liczby naturalnej $n$ nazywamy liczbę
    $$1 \cdot 2 \cdot \ldots \cdot n.$$
    Silnię $n$ oznaczamy przez $n!$.
    Przyjmuje się, że $0!=1$.
    Napisz program, który obliczy silnię danej liczby naturalnej.

    \section{Wejście}
    Na wejściu znajduje się jedna liczba naturalna $n$ ($0 \le n \le 12$).

    \section{Wyjście}
    W jedynym wierszu wyjścia Twój program powinien wypisać wartość $n!$.

    \example{0}


  \end{document}

\documentclass{spiral-kurs}
\def\title{Duże liczby}
\def\id{duz}
\def\TL{1~s}
\def\ML{256~MB}
\begin{document}
\makeheader
%
    Wiemy już, że każdy typ całkowity w języku C++ ma pewien ograniczony zakres.
    W tym zadaniu będziemy jednak operować na liczbach przekraczających
    zakresy wszystkich typów!
    Twoim zadaniem jest napisanie programu, który będzie porównywał takie bardzo duże liczby.
    Jak sobie z tym poradzisz?\ldots

    \section{Wejście}
    Wejście składa się tylko z jednego wiersza, zawierającego kolejno:
    liczbę naturalną $a$, odstęp, znaki porównania, odstęp i liczbę naturalną $b$
    ($1 \le a, b \le 10^{1000}$).
    Możliwe znaki porównania to:
    \texttt{==}, \texttt{!=}, \texttt{<}, \texttt{>}, \texttt{<=} lub \texttt{>=}.

    \section{Wyjście}
    Twój program powinien wypisać jedno słowo \texttt{TAK} lub \texttt{NIE},
    oznaczające, czy nierówność podana na wejściu jest prawdziwa, czy fałszywa.

    \noindent
    Wejście dla testu {\tt \id 0}:
    \includefile{../in/\id 0.in}

    \noindent
    Wyjście dla testu {\tt \id 0}:
    \includefile{../out/\id 0.out}

    \noindent
    natomiast dla danych wejściowych:

    \includefile{../in/\id 0a.in}

    \noindent
    poprawnym wynikiem jest:

    \includefile{../out/\id 0a.out}



  \end{document}

\documentclass{spiral-kurs}
\def\title{Liczby pierwsze}
\def\id{lic}
\def\TL{1~s}
\def\ML{256~MB}
\begin{document}
\makeheader
%
  \emph{Liczbą pierwszą} nazywamy liczbę naturalną, która ma dokładnie dwa dzielniki:
  jedynkę i nią samą.
  Liczby naturalne większe od 1, które nie są liczbami pierwszymi, nazywamy \emph{liczbami złożonymi}.

  W tym zadaniu Twój program powinien stwierdzić, czy wczytana na wejściu liczba naturalna jest
  pierwsza, czy złożona.
    
  \section{Wejście}
  W jedynym wierszu wejścia znajduje się jedna liczba całkowita $n$ ($2 \le n \le 1\,000\,000$).
    
  \section{Wyjście}
  Twój program powinien wypisać jedno słowo \texttt{pierwsza} lub \texttt{zlozona}, określające liczbę $n$.

    \example{0}

    \noindent
    a dla danych wejściowych:

    \includefile{../in/\id 0a.in}

    \noindent
    poprawnym wynikiem jest:
    
    \includefile{../out/\id 0a.out}


  \end{document}

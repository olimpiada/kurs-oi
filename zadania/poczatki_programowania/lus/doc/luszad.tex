\documentclass{spiral-kurs}
\def\title{Lustro}
\def\id{lus}
\def\TL{1~s}
\def\ML{256~MB}
\begin{document}
\makeheader
%
    Zapisem lustrzanym liczby naturalnej $n$ nazywamy liczbę złożoną
    z tych samych cyfr co $n$, tylko w odwrotnej kolejności.
    Oto kilka przykładów liczb oraz ich zapisów lustrzanych.
    Zauważ, że jeśli liczba $n$ ma na końcu zera, to w jej zapisie
    lustrzanym te zera nie występują:

    \begin{center}
      \begin{tabular}{|r|r|}
        \hline
        liczba & zapis lustrzany\\\hline
           123 & 321\\\hline
         55600 & 655\\\hline
             7 & 7 \\\hline
      \end{tabular}
    \end{center}

    \noindent
    Napisz program, który wyznaczy zapis lustrzany danej liczby $n$.

    \section{Wejście}
    Wejście zawiera jedną liczbę naturalną $n$
    ($1 \le n \le 1\,000\,000\,000$).

    \section{Wyjście}
    Jedyny wiersz wyjścia powinien zawierać zapis lustrzany liczby $n$.

    \example{0}


  \end{document}

\documentclass{spiral-kurs}
\def\title{Pomiary}
\def\id{pom}
\def\TL{1~s}
\def\ML{256~MB}
\begin{document}
\makeheader
%
    Czujnik w muzeum w ciągu dnia wykonał serię regularnych pomiarów
    poziomu zanieczyszczenia powietrza w pomieszczeniu.
    Wiadomo, że wizyta każdego zwiedzającego powoduje wzrost zanieczyszczenia powietrza.
    Dyrektor muzeum chciałby oszacować, ilu zwiedzających było tego dnia w muzeum.
    Napisz program, który obliczy, ile istotnie różnych pomiarów o dodatnim poziomie
    zanieczyszczenia zarejestrował czujnik.

    \section{Wejście}
    Wejście składa się z co najmniej dwóch wierszy.
    Każdy wiersz zawiera jedną liczbę całkowitą.
    Pierwszy wiersz zawiera liczbę 0 -- wynik pierwszego pomiaru czujnika.
    Kolejne wiersze zawierają kolejne wyniki pomiarów, będące
    nieujemnymi liczbami całkowitymi.
    Wyniki pomiarów są podane w porządku niemalejącym.
    Ostatni wiersz zawiera liczbę $-1$, oznaczającą koniec wejścia.

    Wejście będzie zawierać co najwyżej $100\,000$ liczb.
    Żadna liczba na wejściu nie przekroczy $1\,000\,000\,000$.

    \section{Wyjście}
    Jedyny wiersz wyjścia powinien zawierać liczbę różnych liczb dodatnich
    występujących na wejściu.
    Jeśli wejście nie zawiera żadnej liczby dodatniej, poprawnym wynikiem jest 0.

    \example{0}


  \end{document}

\documentclass{spiral-kurs}
\def\title{Czas}
\def\id{cza}
\def\TL{1~s}
\def\ML{256~MB}
\begin{document}
\makeheader
%
Napisz program, który przelicza czas podany w sekundach na zapis uwzględniający
godziny, minuty oraz sekundy.

\section{Wejście}
Jedyny wiersz wejścia zawiera jedną liczbę całkowitą $t$ ($1 \le t \le 1\,000\,000$),
oznaczającą czas wyrażony w~sekundach.

\section{Wyjście}
Twój program powinien wypisać czas $t$ w postaci $g\mathtt{g}m\mathtt{m}s\mathtt{s}$, gdzie $g$, $m$ i $s$
oznaczają odpowiednio liczbę godzin, minut i sekund.
Innymi słowy, $g$ godzin, $m$ minut i $s$ sekund powinno łącznie dawać $t$ sekund.

Liczby $g$, $m$ i $s$ powinny być całkowite i nieujemne, a liczby $m$ i $s$ nie powinny przekraczać 59.
W liczbach nie należy wypisywać dodatkowych zer wiodących.

  \example{0}


\end{document}

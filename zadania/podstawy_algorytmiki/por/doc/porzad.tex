\documentclass{spiral-kurs}
\def\title{Portal społecznościowy}
\def\id{por}
\def\TL{15~s}
\def\ML{256~MB}
\begin{document}
\makeheader
%
  
Prowadzisz badania statystyczne dla bajtockiego portalu społecznościowego \texttt{facepalm.bt}. Masz do dyspozycji pełen graf połączeń, czyli wszystkie pary użytkowników, którzy zadeklarowali się jako znajomi (powiemy przy tym, że osoby A i B są \textit{dalszymi znajomymi}, jeśli A może skontaktować się z B prosząc jakiegoś znajomego, aby on poprosił swojego znajomego itd\ldots aby on napisał do B). Ze względu na ochronę danych osobowych podano Ci tylko numery użytkowników, bez nazwisk i innych danych.

Twoi mocodawcy chcieliby wiedzieć, ile jest różnych rozłącznych grup takich, że wewnątrz nich wszyscy są swoimi bliższymi lub dalszymi znajomymi, a między grupami znajomości nie ma. Dodatkowo, sam jesteś ciekaw jak wygląda Twoja grupa i jak długiego łańcucha znajomych potrzebujesz, aby się z każdym skontaktować. Własny numer użytkownika oczywiście znasz.


    \section{Wejście}

Pierwszy wiersz wejścia zawiera liczbę zestawów danych $Z$ -- dla każdego zestawu trzeba osobno obliczyć i podać odpowiedź. Kolejne wiersze zawierają opisy zestawów w następującej postaci:

W pierwszym wierszu zestawu znajdują się dwie liczby naturalne $n$, $m$ ($1 \leq n \leq 200\,000$, $0 \leq m \leq 500\,000$) -- liczba użytkowników portalu i liczba zawartych znajomości. W kolejnych $m$ wierszach znajdują się po dwie liczby naturalne $a$, $b$ ($1 \leq a \neq b \leq n$) -- pary znajomych. Możesz założyć, że każda para wystąpi co najwyżej raz. Ostatni wiersz zestawu zawiera pojedynczą liczbę naturalną -- Twój własny numer użytkownika.
      
    \section{Wyjście}

Dla każdego zestawu wypisz:
\begin{itemize}
\item w pierwszym wierszu ,,Znajomi numeru \textit{(Twój numer)}:'',
\item w kolejnych wierszu Twoich bliższych i dalszych znajomych, w kolejności rosnącej według numerów, w postaci \textit{(numer znajomego)}: \textit{(odległość)}. Odległość powinna wynosić $1$ dla bezpośrednich znajomych, $2$ dla znajomych znajomych itd.
\item w ostatnim wierszu zdanie: ,,Grup znajomych jest \textit{(liczba)}.''
\end{itemize}


    \example{0}
  \end{document}

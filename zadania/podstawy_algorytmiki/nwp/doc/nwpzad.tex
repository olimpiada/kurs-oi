\documentclass{spiral-kurs}
\def\title{Najdłuższy wspólny podciąg}
\def\id{nwp}
\def\TL{0.5~s}
\def\ML{256~MB}
\begin{document}
\makeheader
%

Dane są dwa ciągi znaków. Oblicz długość ich najdłuższego wspólnego podciągu.

    \section{Wejście}

Pierwszy i jedyny wiersz wejścia zawiera dwa ciągi znaków oddzielone spacją. Ciągi składają się z przynajmniej jednego i co najwyżej $2000$ znaków --
małych liter alfabetu angielskiego.


    \section{Wyjście}

Na wyjście należy wypisać jeden wiersz zawierający liczbę całkowitą -- długość szukanego najdłuższego wspólnego podciągu.


    \example{0}
  \end{document}

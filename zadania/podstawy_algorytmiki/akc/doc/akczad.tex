\documentclass{spiral-kurs}
\def\title{Akcelerator}
\def\id{akc}
\def\TL{2~s}
\def\ML{256~MB}
\begin{document}
\makeheader
%
Fizyk-stażysta Bajtazar śledzi działanie Wielkiego Bajtockiego Akceleratora Cząstek. W akceleratorze porusza się duża liczba cząstek o różnych prędkościach (dodatnich albo ujemnych, w zależności od kierunku ruchu).
Zadaniem jest mierzenie tych właśnie prędkości.

Bajtazar wykrył $n$ cząstek i zmierzył ich prędkości. Z braku lepszych zajęć ustawił wszystkie wyniki pomiarów w kolejności niemalejącej.
Opracowanie wyników wymaga jednak odpowiedzi na kilka pytań postaci {\it ,,dla zadanej prędkości, ile jest cząstek, które poruszały się z tą właśnie prędkością?'' }

Pomóż mu znaleźć odpowiedzi i zakończyć staż z pozytywną oceną!

   \section{Wejście}
W pierwszym wierszu wejścia znajduje się liczba całkowita $n$ ($1 \leq n \leq
10^5$) oznaczająca liczbę cząstek. W drugim wierszu
znajduje się $n$ liczb całkowitych o wartości bezwzględnej nie
przekraczającej $10^9$, oddzielonych spacjami --- są to
kolejne prędkości cząstek, uporządkowane niemalejąco.
W trzecim wierszu znajduje się liczba całkowita $q$ ($1 \leq q \leq 10^6$) oznaczająca liczbę
zapytań, które ciekawią Bajtazara. Kolejnych $q$ wierszy zawiera po jednej liczbie całkowitej,
której wartość bezwzględna jest nie większa niż $10^9$ -- są to prędkości, o które pyta Bajtazar.


  \section{Wyjście}
Na wyjście wypisz dokładnie $q$ wierszy. Wiersze te powinny zawierać odpowiedzi na kolejne pytania -- odpowiedzią jest ilość wystąpień podanej liczby wśród odczytów.

\section{Wskazówki}
Nie próbuj przy każdym zapytaniu przejeżdżać pętlą przez wszystkie odczyty, jest to bowiem zbyt czasochłonne i prawie na pewno spowoduje komunikat {\it Przekroczenie limitu czasu}.
Dobrym pomysłem na pierwszy krok jest znalezienie szukanej liczby w tablicy za pomocą wyszukiwania binarnego. Nie powinieneś jednak potem iterować się po elementach tablicy, szukając wszystkich wystąpień tej liczby -- znowu, to również spowoduje, że Twój program będzie działał zbyt długo.
Pamiętaj, że w algorytmice zwykle zakłada się, że Twój program dostanie nieprzyjemne dane wejściowe --- postaraliśmy się zatem o możliwie trudne testy!


  \example{0}

  \end{document}

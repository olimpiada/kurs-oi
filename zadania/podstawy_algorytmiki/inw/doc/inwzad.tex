\documentclass{spiral-kurs}
\def\title{Inwersje}
\def\id{inw}
\def\TL{3~s}
\def\ML{256~MB}
\begin{document}
\makeheader
%

  Dany jest ciąg liczb $a_0, a_1, \ldots, a_{n-1}$. Znajdź liczbę \emph{inwersji} w ciągu -- par elementów $(a_i, a_j)$ takich, że większy stoi przed mniejszym (innymi słowy, $i < j$ oraz $a_i > a_j$).
  Dodatkowo, wypisz elementy ciągu w kolejności rosnącej.

    \section{Wejście}
    W pierwszym wierszu wejścia znajduje się liczba naturalna $n$ -- długość ciągu ($1 \leq n \leq 300 \, 000$). W drugiej -- $n$ liczb
naturalnych $a_0, a_1, \ldots ,a_{n-1}$ ($1 \leq a_i \leq 10^6$). Wszystkie liczby są różne.

    \section{Wyjście}
    W pierwszym wierszu wyjścia wypisz podany ciąg, posortowany rosnąco. W drugim wierszu -- liczbę inwersji, które znajdowały się w ciągu.

    {\bf Uwaga:} Liczba inwersji może być równa nawet $\frac{n(n-1)}{2}$, a to oznacza, że nie zmieści się w typie {\tt int}. Należy do przechowywania tej liczby
    użyć typu {\tt long long}.

    \example{0}
  \end{document}

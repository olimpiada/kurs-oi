\documentclass{spiral-kurs}
\def\title{Naczelny Statystyk Bajtocji}
\def\id{sta}
\def\TL{1~s}
\def\ML{256~MB}
\begin{document}
\makeheader
%


Twoje życie na stanowisku Naczelnego Statystyka Bajtocji niezmiernie
się ostatnimi czasy skomplikowało. Leży przed tobą, w jednym rzędzie, $n$ zeznań podatkowych złożonych przez obywateli, każde wykazujące
pewien zysk, będący liczbą całkowitą (czasem jest to strata, wtedy liczba jest ujemna).
Co chwila wpada jakiś nierozgarnięty Bajtocjanin i chce dokonać korekty swojego zeznania, zmieniając przypisaną mu liczbę.

Na domiar złego, czasem pojawia się przedstawiciel rządu i szuka dziury w całym. Pyta, na przykład:

{\it ,,Panie Statystyku, a proszę mi z tych tutaj podatników pokazać
najbogatszego?'' }

Napisz program, który ułatwi Ci życie na tym jakże odpowiedzialnym stanowisku.

    \section{Wejście}

Pierwszy wiersz wejścia zawiera dwie liczby całkowite $n, m$ ($1 \leq n,m
\leq 200\,000$) będące odpowiednio liczbą zeznań podatkowych
Bajtocjan i liczbą operacji (korekt obywateli lub zapytań polityków) do wykonania.
Następny wiersz zawiera $n$ oddzielonych spacjami liczb całkowitych z przedziału $[-10^6, 10^6]$ -- są to zyski lub straty kolejnych
Bajtocjan.


W kolejnych wierszach znajduje się $m$ poleceń, z których każde składa
się z dwóch wierszy. Polecenia są dwojakiego rodzaju:

\medskip
\texttt{UPDATE}

$k$ $x$
\medskip

oznacza, że $k$-te w kolejności zeznanie Bajtocjanina jest korygowane na
nową wartość $x$ (zeznania numerujemy od $1$-szego do $n$-tego).

\medskip
\texttt{MAX}

$p$ $q$
\medskip

oznacza, że polityk pyta o największe zeznanie spośród Bajtocjan od
$p$-tego do $q$-tego włącznie.



    \section{Wyjście}

Na wyjście wypisz odpowiedzi na wszystkie pytania polityka, w kolejności w jakiej zostały zadane. Każde z nich powinno być pojedynczą liczbą, w osobnym wierszu.

    \example{0}
  \end{document}

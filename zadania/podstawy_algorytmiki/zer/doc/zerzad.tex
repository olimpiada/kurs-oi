\documentclass{spiral-kurs}
\def\title{Zera i jedynki (ale głównie zera)}
\def\id{zer}
\def\TL{1~s}
\def\ML{64~MB}
\begin{document}
\makeheader
%
	Wszystkie dane w pamięci komputera są na pewnym poziomie reprezentowane
	przez ciąg zer i~jedynek. Błędy sprzętowe, skoki napięcia
	i~wysokoenergetyczne cząstki kosmiczne mogą spowodować niespodziewane
	zmiany w~stanie pamięci -- zera zmieniają się na jedynki, a jedynki
	na zera. Twoim zadaniem jest obserwować takie zmiany w~podanym ciągu
	zerojedynkowym i~po~każdej z~nich odpowiadać, jaka jest najdłuższa
	nieprzerwana sekwencja samych zer w~całym ciągu.

	\section{Wejście}
	W~pierwszym wierszu wejścia znajdują się dwie liczby całkowite $n$
	i~$k$ ($5 \leq n \leq 500\,000$, $1 \leq k \leq 200\,000$), oznaczające
	odpowiednio długość ciągu zerojedynkowego i~liczbę zmian w~tym ciągu.
	W~drugim wierszu wejścia znajduje się ciąg $n$ znaków \texttt{0}
	i~\texttt{1}. W~każdym z~kolejnych $k$ wierszy znajduje się jedna liczba
	całkowita z~zakresu od~$1$ do~$n$, oznaczająca pozycję w~podanym ciągu,
	w~której dochodzi to zmiany stanu pamięci

	\section{Wyjście}
	W~kolejnych $k$ wierszach wyjścia wypisz po jednej liczbie całkowitej.
	W~$i$-tym wierszu wypisz długość najdłuższego nieprzerwanego ciągu zer
	powstałego w ciągu wejściowym po~$i$-tej zamianie znaku.

	\vspace{-4pt}
	\example{0}

	\noindent
	\textbf{Podpowiedź:} dla określonego przedziału w ciągu zerojedynkowym
	zapamiętuj jaki jest najdłuższy nieprzerwany ciąg zer od~lewej
	i~od~prawej strony.
\end{document}

\documentclass{spiral-kurs}
\def\title{Rozkład na czynniki}
\def\id{roz}
\def\TL{2~s}
\def\ML{256~MB}
\begin{document}
\makeheader
%

Rozłóż podane liczby na czynniki pierwsze.

    \section{Wejście}
    W pierwszym wierszu wejścia znajduje się liczba naturalna $T \leq 1000$ -- ilość liczb do rozłożenia. W kolejnych wierszach znajdują się liczby,
    każda równa co najmniej $2$ i co najwyżej $10^9$.

    \section{Wyjście}

Dla każdej liczby wypisz jej rozkład na czynniki pierwsze w postaci\\
$n\ =\ a_1 \hat{\ } p_1*a_2\hat{\ }p_2 * \ldots * a_k \hat{\ } p_k$, tak jak w podanym przykładzie. Nie wypisuj wykładnika, jeśli jest równy $1$.

    \example{0}
  \end{document}

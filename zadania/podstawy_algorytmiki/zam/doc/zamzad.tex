\documentclass{spiral-kurs}

  \def\title{Zamiana między systemami}
  \def\id{zam}
  %\def\contest{}
%  \def\desc{Prosta algorytmika / Maksima i podciągi w tablicach}
  \def\TL{5~s}
  \def\ML{256 MB}

\begin{document}
  \makeheader

  Zamień podane liczby z systemu dziesiętnego na inne systemy pozycyjne, lub z innych systemów na dziesiętny.

  \section{Wejście}
Pierwszy wiersz wejścia zawiera liczbę naturalną $Z$ -- liczbę liczb do zamiany. W kolejnych $Z$ wierszach znajdują się kolejno: słowo \texttt{code} lub \texttt{decode}, liczba $x$ do zamiany oraz liczba $b$.

Jeśli słowem jest \texttt{code}, to liczba $x$ jest podana w systemie dziesiętnym i należy ją zapisać w systemie o podstawie $b$.
Jeśli słowem jest \texttt{decode}, to liczba $x$ jest podana w systemie o podstawie $b$ i należy ją zapisać w systemie dziesiętnym.

Liczba $x$ jest zawsze całkowita, nieujemna, a jej wartość nie przekracza $10^9$ w systemie dziesiętnym. Liczba $b$ jest liczbą całkowitą między $2$ a $16$. W systemach o podstawie większej niż $10$ cyfry większe niż $9$ zapisujemy za pomocą kolejnych liter alfabetu: \texttt{A} to $10$, \texttt{B} to $11$,..., \texttt{F} to $16$.

  \section{Wyjście}
Na wyjście Twój program powinien wypisać $Z$ wierszy z odpowiedziami dla kolejnych poleceń. Każda odpowiedź to liczba zamieniona na odpowiedni system pozycyjny.

  \example{0}


\end{document}

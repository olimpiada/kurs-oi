\documentclass{spiral-kurs}

  \def\title{Maksymilian i górskie szczyty}
  \def\id{max}
  %\def\contest{}
%  \def\desc{Prosta algorytmika / Maksima i podciągi w tablicach}
  \def\TL{5~s}
  \def\ML{128 MB}


\begin{document}
  \makeheader
      W dalekiej przyszłości, Maksymilian jest geodetą badającym odległą planetę \emph{Bajtocja V}. Jego zwierzchników, fanów i dziennikarzy interesuje przede wszystkim jedna kwestia: jaki jest najwyższy szczyt na planecie?

      Maksymilian zmierzył już wysokości wszystkich szczytów. Napisz program, który stwierdzi, który z nich jest najwyższy.


  \section{Wejście}
      W pierwszym wierszu wejścia podana jest liczba szczytów $n$ ($1 \leq n \leq 1\,000\,000$). W drugim wierszu znajduje się $n$ liczb całkowitych oddzielonych pojedynczymi odstępami -- wysokości szczytów. Wysokości są dodatnie i nie przekraczają $100\,000$.


  \section{Wyjście}
      Na wyjście Twój program powinien wypisać jedną liczbę całkowitą -- wysokość najwyższego szczytu.

  \example{0}

%   \medskip
%   \noindent\textbf{Wyjaśnienie przykładu:}
%
%   \paragraph{Testy ,,ocen'':}
%   \begin{enumerate}
%   \setlength\itemindent{-13pt}
%
%   \item[] \textbf{\texttt{1ocen:}}
%   \item[] \textbf{\texttt{2ocen:}}
%   \item[] \textbf{\texttt{3ocen:}}
%   \end{enumerate}
%
% \section{Ocenianie}
%
% \begin{center}
%     \begin{tabular}{|c|p{5cm}|c|}
%     \hline
%     \textbf{Podzadanie} & \textbf{Ograniczenia} & \textbf{Punkty} \\
%     \hline
%     \hline
%     \end{tabular}
% \end{center}

\end{document}

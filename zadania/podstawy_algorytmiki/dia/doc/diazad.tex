\documentclass{spiral-kurs}
\def\title{Diamenty}
\def\id{dia}
\def\TL{10~s}
\def\ML{256~MB}
\begin{document}
\makeheader
%

Prostokątna plansza jest podzielona na $h$ wierszy i $w$
kolumn - w związku z tym znajduje się na niej $w*h$ pól. Na niektórych
z tych pól może się znajdować diament (co najwyżej jeden na jedno pole).

Pionek startuje w lewym górnym rogu planszy. Jego zadaniem jest dojście do
prawego dolnego rogu planszy poruszając się wyłącznie 'w dół' lub 'w prawo'.
Innymi słowy pionek przechodzi do dolnego prawego rogu wykonując łącznie $(w-1 + h-1)$ ruchów. W tym czasie pionek
chce zebrać jak najwięcej diamentów.

Twoim zadaniem jest obliczenie ile najwięcej diamentów można łącznie zebrać
podczas podróży.


  \section{Wejście}

W pierwszym wierszu wejścia znajduje się liczba $T$ zestawów testowych do rozwiązania. Potem następują kolejno opisy przypadków testowych, w następującej postaci:

W pierwszym wierszu testu znajdują się dwie liczby $w$ i $h$
($1 \leq w, h \leq 1000$) oznaczające odpowiednio ilość kolumn i wierszy na
planszy.
Kolejne $h$ wierszy zawiera opisy kolejnych wierszy planszy. Każdy
wiersz opisany jest przez $w$ liczb całkowitych pooddzielanych spacjami,
o wartości \verb"1" gdy na danym polu znajduje się diament lub \verb"0" gdy diamentu nie ma.
Wiersze podawane są od góry do dołu, natomiast pola wewnątrz wiersza od lewej
do prawej.

  \section{Wyjście}
Każdemu zestawowi odpowiada dokładnie jeden wiersz wyjścia. W wierszu tym
znajduje się pojedyncza liczba całkowita oznaczająca maksymalną liczbę
diamentów, które można zebrać podczas podróży.



    \example{0}


  \end{document}

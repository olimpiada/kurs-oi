\documentclass{spiral-kurs}
\def\title{Piramida}
\def\id{pir}
\def\TL{1~s}
\def\ML{256~MB}
\begin{document}
\makeheader
%

I oto nadszedł jakże smutny dzień, w którym faraon (oby żył wiecznie!) odszedł z tego świata. Dla uczczenia jego pamięci wybudowano, a jakże, wielką piramidę, gdzie złożono jego mumię, a także -- jak powiadają -- nieprzebrane skarby.
Nowy faraon (oby żył wiecznie!) poprowadził uroczystości żałobne i~życie potoczyło się dalej\ldots

W jakiś czas później Pteppic, najzdolniejszy z miejscowych złodziei, postanowił na własne oczy przekonać się, ile jest prawdy w legendzie o~skarbach faraona. Wziął więc swój ulubiony plecak (o~którym jeszcze usłyszymy), wierny komplet wytrychów i~ruszył na podbój świata w ogólności, a piramidy w konkretach. Po pokonaniu wielu zabezpieczeń i pułapek dotarł do ostatniego korytarza, wyłożonego kamiennymi płytami. W tym miejscu architekt piramidy sięgnął granic geniuszu -- żadna z płyt (których jest $n$, numerowanych od $1$ do $n$) nie zdradzała nawet najdrobniejszym szczegółem, że mogłaby kryć śmiercionośną pułapkę. Pteppic postanowił jednak, że nie wycofa się tak blisko celu.
Powierzając swój los egipskim bogom wyciągnął ulubioną kostkę\ldots

Pteppic stoi na płycie numer $1$ (na której, jak już wie, pułapki nie ma). Rzuca kością sześcienną, po czym przeskakuje o tyle płyt do przodu, ile oczek wypadło na kości. Jeśli trafi na pułapkę, jego przygoda kończy się w sposób szybki, acz bezbolesny (architekt był wprawdzie geniuszem, ale humanitarnym). Jeśli trafi na pole numer $n$, skarb jest jego. Jeśli zaś przeskoczy za daleko, pozna ukrytą w przeciwległej ścianie jeszcze jedną niespodziankę architekta, równie skuteczną jak poprzednie.

Na ile różnych sposobów Pteppic może dostać się z pola $1$ na pole $n$? Tak się składa, że architekt jest Twoim znajomym i znasz dobrze położenie pułapek. Odpowiedź może być dużą liczbą, wystarczy zatem, jeśli podasz resztę z dzielenia modulo $k$.

    \section{Wejście}

W pierwszym wierszu wejścia znajdują się liczby naturalne $n$, $k$
($2 \leq n \leq 1\,000\,000$, $2 \leq k \leq 1\,000\,000$), oddzielone pojedynczym
odstępem.
Drugi wiersz zawiera ciąg $n$ znaków \texttt{'0'} i \texttt{'1'}, bez odstępów między nimi.
$0$ na $i$-tej pozycji oznacza, że na $i$-tym polu jest pułapka.
Możesz założyć, że pola $1$ i $n$ są bezpieczne.


    \section{Wyjście}

Na wyjście należy wypisać jeden wiersz zawierający liczbę całkowitą -- liczbę
sposobów (modulo $k$), na jaką można dojść z pola $1$ do pola $n$ pozostając żywym.


    \example{0}
  \end{document}

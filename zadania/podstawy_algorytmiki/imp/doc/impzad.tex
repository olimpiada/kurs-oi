\documentclass{spiral-kurs}
\def\title{Impreza}
\def\id{imp}
\def\TL{1~s}
\def\ML{64~MB}
\begin{document}
\makeheader
%
Chcesz zorganizować dużą imprezę dla swoich znajomych. Niestety niektórzy z
nich się nie lubią --- zaproszenie nielubiącej się pary osób do wspólnej zabawy
zakończy się popsuciem atmosfery na imprezie wszystkim gościom.

Wiesz którzy znajomi się nie lubią. Chcesz wybrać taki podzbiór znajomych, aby
był on jak największy (im huczniejsza impreza, tym lepiej!) oraz aby wszyscy
znajomi w tym podzbiorze się lubili.

\section{Wejście}
W~pierwszym wierszu standardowego wejścia znajduje się liczba $N$ ($3 \leq
N \leq 18$), oznaczająca liczbę znajomych.

W~$i$-tym z~kolejnych $N$ wierszy znajduje się opis relacji $i$-tego
znajomego w~postaci ciągu~$N$ znaków \texttt{0} i~\texttt{1}. $j$-ty z~tych
znaków oznacza w~jakich stosunkach pozostają znajomi $i$, $j$ --- jeśli
\texttt{0}, to się lubią, a~jeśli \texttt{1}, to się nie lubią. Można przyjąć,
że każda osoba lubi samą siebie oraz że jeśli dana osoba lubi (lub nie) inną,
to tamta odwzajemnia uczucie.

\section{Wyjście}
W~pierwszym wierszu standardowego wejścia wypisz jedną liczbę, oznaczającą ile
najwięcej lubiących się znajomych można zaprosić na~imprezę. W~drugim wierszu
wypisz listę tych znajomych w~postaci liczb z~przedziału od~$1$ do~$N$,
oddzielonych pojedynczą spacją. Jeśli jest wiele poprawnych odpowiedzi, wypisz
leksykograficznie minimalną.
	\vspace{-4pt}
	\example{0}
\end{document}

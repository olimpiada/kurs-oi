\documentclass{spiral-kurs}
\def\title{Smakołyki}
\def\id{sma}
\def\TL{10~s}
\def\ML{256~MB}
\begin{document}
\makeheader
%
    Natalia ustawiła w rzędzie $n$ smakołyków. Każdy smakołyk ma przypisany pewien rodzaj.
    Natalia może teraz wybrać pewną liczbę (od $1$ do $n$) sąsiednich smakołyków, a następnie je wszystkie zjeść.
    Jedynym warunkiem jest to, aby żadne dwa smakołyki nie były tego samego rodzaju.
    Pomóż Natalii i znajdź liczbę sposobów, na które może wybrać sąsiednie smakołyki.

    \section{Wejście}
    Pierwszy wiersz wejścia zawiera dwie liczby całkowite $n, m$ ($1 \leq n, m \leq 1\,000\,000$),
    oznaczające odpowiednio liczbę smakołyków oraz liczbę dostępnych ich rodzajów.
    Drugi wiersz zawiera $n$ liczb całkowitych $c_0, c_1, \ldots, c_{n-1}$ ($1 \leq c_i \leq m$),
    gdzie $c_i$ oznacza rodzaj $i$-tego smakołyka.

    \section{Wyjście}
    Pierwszy i jedyny wiersz wyjścia powinien zawierać jedną liczbę całkowitą,
    równą liczbie sposobów, na które Natalia może wybrać sąsiednie smakołyki.

    \example{0}

  \end{document}

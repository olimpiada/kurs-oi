\documentclass{spiral-kurs}
\def\title{Skarb faraona}
\def\id{ska}
\def\TL{15~s}
\def\ML{256~MB}
\begin{document}
\makeheader
%
Uniknąwszy pułapek, zdolny rabuś Pteppic znalazł się w skarbcu piramidy faraona. Skarbów okazało się tam być dość, aby wynagrodzić poprzednie niebezpieczeństwa. Jest jednak jedno ,,ale'': jeśli Pteppic obciąży się za bardzo, może nie być w stanie przeskoczyć nad jakąś pułapką w drodze powrotnej, czego zdecydowanie by nie chciał. Woli pozostać nieco biedniejszy, za to trochę bardziej żywy\ldots

A zatem, jest do wyboru $n$ przedmiotów, z których każdy ma swoją wagę $s_i$ i wartość $v_i$. Do swojego ulubionego plecaka Pteppic może zapakować przedmioty o łącznej wadze nie przekraczającej $p$. Jaka jest największa wartość tego, co może zarobić?

  \section{Wejście}

W pierwszym wierszu wejścia znajduje się liczba $T$ przypadków testowych do rozwiązania. Potem następują kolejno opisy przypadków testowych, w następującej postaci:

W pierwszym wierszu testu znajdują się dwie dodatnie liczby całkowite $n$ i $p$ ($0\leq n \leq 1000$, $1 \leq p \leq 10000$) oddzielone spacją, oznaczające odpowiednio liczbę przedmiotów w skarbcu i pojemność plecaka. W kolejnych $n$ liniach znajdują się opisy przedmiotów; $i$-ta linia składa się z dwóch liczb całkowitych $s_i$ i $v_i$ ($1 \leq s_i \leq 10000$, $1 \leq v_i \leq 10^6$) oddzielonych spacją, oznaczających odpowiednio wagę i wartość $i$-tego przedmiotu.


  \section{Wyjście}


Dla każdego testu wypisz w osobnym wierszu jedną liczbę -- największą możliwą wartość zabranych przedmiotów.

    \example{0}


  \end{document}

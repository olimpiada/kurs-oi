\documentclass{spiral-kurs}
\def\title{Zagadka Nicolo Tartaglii}
\def\id{tar}
\def\TL{1~s}
\def\ML{256~MB}
\begin{document}
\makeheader
%
   W XVI wieku dwóch utalentowanych matematyków -- Gerolamo Cardano i Nicolo Tartaglia -- prowadziło spór o to, kto pierwszy z nich nauczył się rozwiązywać równania trzeciego stopnia (takie jak $x^3 + 3x = 14$).
   Chociaż prawdopodobnie rację miał Tartaglia, w publicznym sporze jednak zwyciężył Cardano, którego nazwiskiem nazwano odpowiednie wzory. Pewne znaczenie miał tu fakt, że Tartaglia przegrał (modny w owych czasach) 
pojedynek na zadania matematyczne, który prowadził z uczniem Cardana, Lodovico Ferrarim.
   W tym pojedynku -- jak sądzimy -- obaj zadawali sobie nawzajem równania do rozwiązania, lecz Ferrari opanował tę sztukę znacznie sprawniej, a ponadto umiał rozwiązywać nawet trudniejsze równania, czwartego stopnia.

   Jak Ty sprawiłbyś się w pojedynku z Tartaglią? Jego zagadki -- ze względu na niechęć do używania w owych czasach liczb ujemnych -- sformułowane są następująco: \\\

   { \it
   Mając dane liczby naturalne $p$ i $q$, znajdź taką liczbę naturalną $x$, dla której

	$$x^3 + px = q$$

   lub ustal, że taka liczba nie istnieje.
   }

   \section{Wejście}
   W pierwszym wierszu wejścia znajduje się liczba zagadek $z$, nie większa niż $10\,000$. W kolejnych wierszach podane są zagadki, z których każda to dwie liczby naturalne $p$ i $q$. Obie liczby są dodatnie, liczba $p$ nie przekracza $10^{12}$, zaś $q$ -- $10^{18}$.
Liczby są oddzielone pojedynczym odstępem, każda zagadka podana jest w~osobnym wierszu.
  \section{Wyjście}
  Wypisz odpowiedzi na wszystkie zagadki, w tej kolejności, w jakiej były podane na wejściu. Odpowiedź powinna być liczbą naturalną spełniającą podane równanie. Jeśli taka liczba nie istnieje, zamiast liczby wypisz słowo {\tt NIE}.

  \section{Wskazówki}
  Chociaż istnieją wzory na rozwiązanie takiego równania (zwane {\it wzorami Cardana}) użycie ich w programie może okazać się kłopotliwe. Prościej jest użyć wyszukiwania binarnego. 
Zauważ też, że podane na wejściu liczby są za duże dla zmiennych typu {\tt int} -- użyj typu {\tt long long}.

  \example{0}

  \end{document}

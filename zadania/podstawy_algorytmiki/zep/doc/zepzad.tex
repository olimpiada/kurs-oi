\documentclass{spiral-kurs}
\def\title{Zeznania podatkowe}
\def\id{zep}
\def\TL{10~s}
\def\ML{256~MB}
\begin{document}
\makeheader
%
W Bajtocji dobiegł właśnie końca kolejny rok podatkowy. Zeznania podatkowe obywateli leżą przed Tobą, jeszcze bez żadnego porządku. Nowy rząd chce od Ciebie odpowiedzi na zupełnie inne pytanie: zapytany o pewną grupę Bajtocjan tworzącą spójny przedział (czyli od $a$-tego do $b$-tego Bajtocjanina włącznie, dla pewnych $a$ i $b$) powiedz, ile wynosi ich łączny zysk.

   \section{Wejście}
W pierwszym wierszu wejścia znajduje się liczba całkowita $n$ ($1 \le n \le 10^6$) oznaczajaca liczbę zeznań podatkowych.
W drugim wierszu znajduje się $n$ liczb całkowitych dodatnich o wartości nie przekraczającej $10^9$ oddzielonych spacjami -- są zyski kolejnych Bajtocjan.
W trzecim wierszu znajduje się liczba całkowita $q$ ($1 \le q \le 10^6$) oznaczająca liczbę zapytań. Kolejnych $q$ wierszy zawiera po dwie liczby całkowite $a$, $b$ ($1 \leq a \leq b \leq n$).


  \section{Wyjście}
Dla każdego zestawu danych wypisz $q$ linii zawierających odpowiedzi na zadane pytania.
Odpowiedzią jest łączny zysk Bajtocjan od $a$-tego do $b$-tego włącznie.


  \example{0}

  \end{document}

\documentclass{spiral-kurs}
\def\title{Randka w ciemno}
\def\id{ran}
\def\TL{6~s}
\def\ML{256~MB}
\begin{document}
\makeheader
%
    Najpopularniejszym ostatnio w Bajtocji telewizyjnym \textit{show} jest ,,Randka w ciemno''. Uczestnikom programu rozdaje się kartki z wylosowanymi dla nich liczbami naturalnymi, po czym losuje się jeszcze jedną liczbę naturalną $s$.
    Jeśli dwoje Bajtocjan ma kartki z liczbami dającymi w sumie $s$,
    wygrywa wycieczkę życia (po której, być może, następuje trochę zamieszania, aż wszyscy dojdą do wniosku, że będą żyć długo i szczęśliwie).
    Tobie przypadła w udziale obsługa systemu komputerowego telewizji. Wiadomo, jakie karty rozdano i jaką liczbę wylosowano -- kto, według Ciebie, zgłosi się po wygraną?

    \section{Wejście}
    W pierwszym wierszu wejścia znajdują się dwie liczby naturalne $n$ i $s$ --- odpowiednio liczba uczestników ($1 \leq n \leq 200\,000$) i wylosowana liczba ($1 \leq s \leq 2\cdot 10^9$).
    W każdym z kolejnych $n$ wierszy znajduje się imię Bajtocjanina (lub Bajtocjanki) oraz wręczona mu (lub jej) liczba. Imiona nie przekraczają $10$ znaków długości i składają się z małych lub wielkich liter alfabetu angielskiego.
    Liczby rozdawane uczestnikom są dodatnie i nie większe niż $10^9$. Dodatkowo, liczba $s$ jest zawsze nieparzysta.
    \section{Wyjście}
    Dla każdego zestawu, jeśli nikt nie wygra w tej edycji, wypisz pojedyncze słowo \texttt{NIE}, w przeciwnym wypadku trzeba wypisać oddzielone spacją imiona zwycięskich Bajtocjan. Jeśli istnieje więcej niż jedna możliwość, wypisz dowolną z nich.
    \section{Wskazówka}
    Posortuj uczestników względem otrzymanej przez nich liczby. Weź pierwszego z nich i zastanów się, czy można do niego dobrać właściwą parę, oraz jak znaleźć ją w tablicy uczestników.

    \example{0}
  \end{document}
